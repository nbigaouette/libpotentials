%%%%%%%%%%%%%%%%%%%%%%%%%%%%%%%%%%%%%%%%%%%%%%%%%%%%%%%%%%
%%%%%%%%%%%%%%%%%%%%%%%%%%%%%%%%%%%%%%%%%%%%%%%%%%%%%%%%%%
%%          Written by Nicolas Bigaouette               %%
%%                    Winter 2008                       %%
%%              nbigaouette@gmail.com                   %%
%%%%%%%%%%%%%%%%%%%%%%%%%%%%%%%%%%%%%%%%%%%%%%%%%%%%%%%%%%
%%%%%%%%%%%%%%%%%%%%%%%%%%%%%%%%%%%%%%%%%%%%%%%%%%%%%%%%%%

\documentclass[12pt,letterpaper]{article}
% \documentclass[12pt,letterpaper,draft]{article}

\title{Expansions}
\date{\today}
% \date{Wednesday, November 4th (\today)}
\author{Nicolas Bigaouette}

%%%%%%%%%%%%%%%%%%%%%%%%%%%%%%%%%%%%%%%%%%%%%%%%%%%%%%%
% Useful macros
%%%%%%%%%%%%%%%%%%%%%%%%%%%%%%%%%%%%%%%%%%%%%%%%%%%%%%%
\input{macros.tex}
%%%%%%%%%%%%%%%%%%%%%%%%%%%%%%%%%%%%%%%%%%%%%%%%%%%%%%%


%%%%%%%%%%%%%%%%%%%%%%%%%%%%%%%%%%%%%%%%%%%%%%%%%%%%%%%
% Needed packages
%%%%%%%%%%%%%%%%%%%%%%%%%%%%%%%%%%%%%%%%%%%%%%%%%%%%%%%

% Needed for \begin{subequations}...\end{subequations}
\usepackage{amsmath}
% Needed for bold Greek letters
\usepackage{bm}
% Multiple figures
% \usepackage{subfigure}
% For 1/2 symbol
\usepackage{ltugcomn}
% For \cqfd
% \usepackage{latexsym}
% To prevent a newline after call to ``\subsection{}''
% See http://stackoverflow.com/questions/1469096/no-newline-after-subsection
% \usepackage{titlesec}
% \titleformat{\subsection}[runin]{\bfseries}{}{}{}[]
% \usepackage{empheq}
%%%%%%%%%%%%%%%%%%%%%%%%%%%%%%%%%%%%%%%%%%%%%%%%%%%%%%%
\usepackage{ifthen}
%%%%%%%%%%%%%%%%%%%%%%%%%%%%%%%%%%%%%%%%%%%%%%%%%%%%%%%


% \usepackage{lscape}
% \usepackage{pdflscape}
% \def\Tiny{\fontsize{6pt}{6pt}\selectfont}


%%%%
% Voir http://forums.gentoo.org/viewtopic-t-202973-highlight-fonts+latex.html
% \usepackage{mathptmx}
% \usepackage[scaled=.90]{helvet}
% \usepackage{courier}
%%%%

% cd /tmp
% wget http://ftp.ktug.or.kr/tex-archive/macros/latex/contrib/misc/cases.sty
% sudo mkdir /usr/share/texmf/tex/latex/cases/
% sudo cp test/cases.sty /usr/share/texmf/tex/latex/cases/
% sudo texhash
%\usepackage{cases}                  %
% http://ftp.ktug.or.kr/tex-archive/macros/latex/contrib/misc/cases.sty

\usepackage{ifpdf}
\ifpdf
    %%%%%%%%%%%%%%%%%%%%%%%%%%%%%%%%%%%%%%%%%%%%%%%%%%%%%%%
    % PDF compilation with "pdflatex"
    %%%%%%%%%%%%%%%%%%%%%%%%%%%%%%%%%%%%%%%%%%%%%%%%%%%%%%%
    \usepackage[pdftex]{graphicx}
    \def\pdfshellescape{1}
    \pdfcompresslevel=9
    %%%
    % PDF options
    % See http://barrault.free.fr/ressources/rapports/pdflatex/
    \usepackage[pdftex,
        bookmarks = true,
        bookmarksnumbered = true,
        bookmarksopen = false,
        %pdfpagemode = UseOutlines,
        pdfpagemode = UseNone,
        pdfstartview = FitH,
        colorlinks,
        citecolor=black,urlcolor=blue,linkcolor=black,
        pdfauthor={Nicolas Bigaouette},
        pdftitle={Expansions},
        unicode = true,
        plainpages = false,pdfpagelabels
    ]{hyperref}
\else
    %%%%%%%%%%%%%%%%%%%%%%%%%%%%%%%%%%%%%%%%%%%%%%%%%%%%%%%
    % DVI compilation with "latex"
    %%%%%%%%%%%%%%%%%%%%%%%%%%%%%%%%%%%%%%%%%%%%%%%%%%%%%%%
    \usepackage[dvips]{graphicx}
    \newcommand{\url}[1]{{\color{blue}#1}}
\fi

%%%%%%%%%%%%%%%%%%%%%%%%%%%%%%%%%%%%%%%%%%%%%%%%%%%%%%%%%%
%                       Page format
%%%%%%%%%%%%%%%%%%%%%%%%%%%%%%%%%%%%%%%%%%%%%%%%%%%%%%%%%%
\textheight=21cm
\textwidth=17.0cm
\oddsidemargin=0cm
\evensidemargin=0cm
\topmargin=0cm
\headsep=20pt
\topskip=10pt
\large
%%%%%%%%%%%%%%%%%%%%%%%%%%%%%%%%%%%%%%%%%%%%%%%%%%%%%%%%%%

% These adjust how LaTeX puts figures onto pages with text. These values
% reduce the likelihood that a figure will end up by itself on a page.
\renewcommand{\topfraction}{0.85}
\renewcommand{\textfraction}{0.1}
\renewcommand{\floatpagefraction}{0.75}


\begin{document}
\maketitle
% \osvpace{-50 pts}


% \tableofcontents
% \newpage
% *************************************************************

\section{Expansions}
The potential and field of a charge distribution are:
\begin{align}
\phi\pa{\vr} & = \frac{k Q}{r} \erf{\frac{r}{\sqrt{2} \sigma}}
\\
\vE\pa{\vr}  & = k Q \pa{
    \frac{ \erf{\frac{r}{\sqrt{2} \sigma}} }{r^2}
    - \sqrt{ \frac{2}{\pi} } \frac{ \ex{ -\frac{r^2}{2 \sigma^2} } }{\sigma r}
} \orv
\end{align}
Because $r$ might become quite small if 2 particles are close, $\orv$ can cause problem.
We thus replace it with $\orv = \vr / \abs{\vr}$:
\begin{align}
\vE\pa{\vr}  & = k Q \pa{
    \frac{ \erf{\frac{r}{\sqrt{2} \sigma}} }{r^2}
    - \sqrt{ \frac{2}{\pi} } \frac{ \ex{ -\frac{r^2}{2 \sigma^2} } }{\sigma r}
} \frac{\vr}{r}
\\
\vE\pa{\vr}  & = k Q \pa{
    \frac{ \erf{\frac{r}{\sqrt{2} \sigma}} }{r^3}
    - \sqrt{ \frac{2}{\pi} } \frac{ \ex{ -\frac{r^2}{2 \sigma^2} } }{\sigma r^2}
} \vr
\end{align}
The value $\vr$ is easy to calculate. For two particle $p_0$ and $p_1$, it is $\vr = \vr_0 - \vr_1$.

Using $\orv = \vr / \abs{\vr}$ and $x = \frac{r}{\sqrt{2} \sigma}$, the previous equations
are simplified to:
\begin{align}
\vE\pa{x}
& = k Q \pa{
    \frac{ \erf{x} }{\pa{\sqrt{2} \sigma x}^3}
    - \sqrt{ \frac{2}{\pi} } \frac{ \ex{-x^2} }{\sigma \pa{\sqrt{2} \sigma x}^2}
} \vr
\nonumber \\
& = k Q \pa{
    \frac{ \erf{x} }{2^{3/2} \sigma^3 x^3}
    - \sqrt{ \frac{2}{\pi} } \frac{ \ex{-x^2} }{2 \sigma^3 x^2}
} \vr
\nonumber \\
\vE\pa{x}
& = \frac{k Q}{\sqrt{2} \sigma^3} \pa{
    \frac{ \erf{x} }{2 x^3}
    - \frac{1}{\sqrt{\pi}} \frac{ \ex{-x^2} }{x^2}
} \vr
\nonumber \\
\vE\pa{x}
& = \frac{k Q}{\sqrt{2} \sigma^3} F\pa{x} \vr
\\
\phi\pa{x} & = \frac{k Q}{\sqrt{2} \sigma} \frac{\erf{x}}{x}
\\
\phi\pa{x} & = \frac{k Q}{\sqrt{2} \sigma} G\pa{x}
\end{align}
where
\begin{align}
G\pa{x} & = \frac{\erf{x}}{x} \\
F\pa{x} & = \frac{ \erf{x} }{2 x^3} - \frac{1}{\sqrt{\pi}} \frac{ \ex{-x^2} }{x^2}
\end{align}

A lookup table will be used for $F\pa{x}$ and $G\pa{x}$, and for $x<1$, an expansion is
performed. This is necessary since $F\pa{x}$ is prone to floating point errors
for $x \rightarrow 0$ ($0/0 - 1/0$).

Wolfram Alpha gives the expansion. $G\pa{x}$ now
becomes\footnote{\url{http://www.wolframalpha.com/input/?i=erf\%28x\%29\%2Fx}}:
\begin{align}
\sqrt{\pi} G\pa{x} & \simeq
      2
    - \frac{2 x^{2}}{3}
    + \frac{x^{4}}{5}
    - \frac{x^{6}}{21}
    + \frac{x^{8}}{108}
    - \frac{x^{10}}{660}
    + \frac{x^{12}}{4680}
    - \frac{x^{14}}{37800}
    + \frac{x^{16}}{342720}
    - \frac{x^{18}}{3447360}
    + \frac{x^{20}}{38102400}
\nonumber \\ & \hspace{22pt}
    - \frac{x^{22}}{459043200}
    + \frac{x^{24}}{5987520000}
    - \frac{x^{26}}{84064780800}
    + \frac{x^{28}}{1264085222400}
    - \frac{x^{30}}{20268952704000}
\nonumber \\ & \hspace{22pt}
    + \frac{x^{32}}{345226033152000}
    - \frac{x^{34}}{6224529991680000}
    + \frac{x^{36}}{118443913555968000}
\nonumber \\ & \hspace{22pt}
    - \frac{x^{38}}{2372079457972224000}
    + \frac{x^{40}}{49874491167621120000}
    + O\pa{x^{41}}
\end{align}
and $F\pa{x}$
is\footnote{\url{http://www.wolframalpha.com/input/?i=expansion+erf\%28x\%29\%2F\%282*x**3\%29-1\%2Fsqrt\%28pi\%29*exp\%28-x**2\%29\%2F\%28x**2\%29}}:
\begin{align}
\sqrt{\pi} F\pa{x} & \simeq
      \frac{2}{3}
    - \frac{2 x^{2}}{5}
    + \frac{x^{4}}{7}
    - \frac{x^{6}}{27}
    + \frac{x^{8}}{132}
    - \frac{x^{10}}{780}
    + \frac{x^{12}}{5400}
    - \frac{x^{14}}{42840}
    + \frac{x^{16}}{383040}
    - \frac{x^{18}}{3810240}
    + \frac{x^{20}}{41731200}
\nonumber \\ & \hspace{24pt}
    - \frac{x^{22}}{498960000}
    + \frac{x^{24}}{6466521600}
    - \frac{x^{26}}{90291801600}
    + \frac{x^{28}}{1351263513600}
    - \frac{x^{30}}{21576627072000}
\nonumber \\ & \hspace{24pt}
    + \frac{x^{32}}{366148823040000}
    - \frac{x^{34}}{6580217419776000}
    + \frac{x^{36}}{124846287261696000}
\nonumber \\ & \hspace{24pt}
    - \frac{x^{38}}{2493724558381056000}
    + \frac{x^{40}}{52307393175797760000}
\nonumber \\ & \hspace{24pt}
    - \frac{x^{42}}{1149546198863462400000}
    + \frac{x^{44}}{26414017102773780480000}
    +O\pa{x^{45}}
\end{align}

Figure \ref{fig:expansions_F_G} shows the expansions of $F\pa{x}$ and $G\pa{x}$
compared to the exact values. For $x < 1.5$, the error is less then the floating
point noise.

\begin{figure}
    \includegraphics[width=0.98\columnwidth]{expansions_F_G}
    \caption{Expansions of $F\pa{x}$ and $G\pa{x}$}
    \label{fig:expansions_F_G}
\end{figure}


\section{Expansions with r=log(a)}
Let's define:
\begin{align}
a &= \lnp{x} \\
\e{a} & = x
\end{align}
The two functions $F$ and $G$ becomes:
\begin{align}
G\pa{x} & = \frac{\erf{\e{a}}}{\e{a}} \\
F\pa{x} & = \frac{ \erf{\e{a}} }{2 \pa{\e{a}}^3} - \frac{1}{\sqrt{\pi}} \frac{ \ex{-\pa{\e{a}}^2} }{\pa{\e{a}}^2} \\
F\pa{x} & = \frac{ \erf{\e{a}} }{2 \e{3 a}} - \frac{1}{\sqrt{\pi}} \frac{ \ex{-\e{2 a}} }{\e{2 a}}
\end{align}

The expansions are:
\begin{align}
% http://www.wolframalpha.com/input/?i=erf%28exp%28a%29%29+%2F+%28exp%28a%29%29
G\pa{x} & \approx
    erf(1)
    +a (2/(e sqrt(pi))-erf(1))
    +a^2 ((erf(1))/2-3/(e sqrt(pi)))
    +a^3 (1/(e sqrt(pi))-(erf(1))/6)
    +1/24 a^4 (erf(1)+10/(e sqrt(pi)))
    +1/120 a^5 (22/(e sqrt(pi))-erf(1))
    +1/720 a^6 (erf(1)-150/(e sqrt(pi)))
    -(a^7 (e erf(1)+1002/sqrt(pi)))/(5040 e)
    +(a^8 (erf(1)-1302/(e sqrt(pi))))/40320
    +(a^9 (26902/(e sqrt(pi))-erf(1)))/362880
    +(a^10 (e erf(1)+246506/sqrt(pi)))/(3628800 e)
    +(a^11 (599318/sqrt(pi)-e erf(1)))/(39916800 e)
    -(a^12 (9528598/sqrt(pi)-e erf(1)))/(479001600 e)
    -(a^13 (e erf(1)+135723754/sqrt(pi)))/(6227020800 e)
    -(a^14 (692208918/sqrt(pi)-e erf(1)))/(87178291200 e)
    +(a^15 (4302456086/sqrt(pi)-e erf(1)))/(1307674368000 e)
    +(a^16 (e erf(1)+124593552106/sqrt(pi)))/(20922789888000 e)
    +(a^17 (1178087671062/sqrt(pi)-e erf(1)))/(355687428096000 e)
    +(a^18 (e erf(1)+1086500420330/sqrt(pi)))/(6402373705728000 e)
    -(a^19 (e erf(1)+147087633201898/sqrt(pi)))/(121645100408832000 e)
    -(a^20 (2517298379179286/sqrt(pi)-e erf(1)))/(2432902008176640000 e)
    -(a^21 (e erf(1)+18070149954321130/sqrt(pi)))/(51090942171709440000 e)
    +(a^22 (e erf(1)+132038987875400426/sqrt(pi)))/(1124000727777607680000 e)
    +(a^23 (5953921562136495382/sqrt(pi)-e erf(1)))/(25852016738884976640000 e)
    +(a^24 (e erf(1)+87873383278392163050/sqrt(pi)))/(620448401733239439360000 e)
    +(a^25 (443078107750850315542/sqrt(pi)-e erf(1)))/(15511210043330985984000000 e)
    -(a^26 (11743714906826509075734/sqrt(pi)-e erf(1)))/(403291461126605635584000000 e)
    -(a^27 (e erf(1)+370523876199881176562410/sqrt(pi)))/(10888869450418352160768000000 e)
    -(a^28 (5218119697123425490257174/sqrt(pi)-e erf(1)))/(304888344611713860501504000000 e)
    -(a^29 (e erf(1)+15610098398647240048556778/sqrt(pi)))/(8841761993739701954543616000000 e)
    +(a^30 (e erf(1)+1220358267742920850482969322/sqrt(pi)))/(265252859812191058636308480000000 e)
    +(a^31 (35791740664812835109132518678/sqrt(pi)-e erf(1)))/(8222838654177922817725562880000000 e)
    +O(a^32)
\end{align}



% *************************************************************


% % *************************************************************
% \label{references}
% \newpage
% % \nocite{*}    % Si commenté, n'inclut pas ce qui n'est pas cité
% % \bibliographystyle{iso690-2}
% \bibliographystyle{plain}
% % \bibliographystyle{ametsoc}
% \bibliography{references}
% % \bibliography{library}



\end{document}
