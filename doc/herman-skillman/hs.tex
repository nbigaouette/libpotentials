%%%%%%%%%%%%%%%%%%%%%%%%%%%%%%%%%%%%%%%%%%%%%%%%%%%%%%%%%%
%%%%%%%%%%%%%%%%%%%%%%%%%%%%%%%%%%%%%%%%%%%%%%%%%%%%%%%%%%
%%          Written by Nicolas Bigaouette               %%
%%                    Winter 2008                       %%
%%              nbigaouette@gmail.com                   %%
%%%%%%%%%%%%%%%%%%%%%%%%%%%%%%%%%%%%%%%%%%%%%%%%%%%%%%%%%%
%%%%%%%%%%%%%%%%%%%%%%%%%%%%%%%%%%%%%%%%%%%%%%%%%%%%%%%%%%

\documentclass[12pt,letterpaper]{article}
% \documentclass[12pt,letterpaper,draft]{article}
\newcommand{\mytitle}{Herman Skillman Potential Notes}
\title{\mytitle}
\date{}
\author{Nicolas Bigaouette}

%%%%%%%%%%%%%%%%%%%%%%%%%%%%%%%%%%%%%%%%%%%%%%%%%%%%%%%
% Useful macros
%%%%%%%%%%%%%%%%%%%%%%%%%%%%%%%%%%%%%%%%%%%%%%%%%%%%%%%
\input{macros.tex}
%%%%%%%%%%%%%%%%%%%%%%%%%%%%%%%%%%%%%%%%%%%%%%%%%%%%%%%


%%%%%%%%%%%%%%%%%%%%%%%%%%%%%%%%%%%%%%%%%%%%%%%%%%%%%%%
% Needed packages
%%%%%%%%%%%%%%%%%%%%%%%%%%%%%%%%%%%%%%%%%%%%%%%%%%%%%%%

% Needed for \begin{subequations}...\end{subequations}
\usepackage{amsmath}
% Needed for bold Greek letters
\usepackage{bm}
% Multiple figures
\usepackage{subfigure}
%%%%%%%%%%%%%%%%%%%%%%%%%%%%%%%%%%%%%%%%%%%%%%%%%%%%%%%
\usepackage{ifthen}
%%%%%%%%%%%%%%%%%%%%%%%%%%%%%%%%%%%%%%%%%%%%%%%%%%%%%%%

%%%%
% Voir http://forums.gentoo.org/viewtopic-t-202973-highlight-fonts+latex.html
% \usepackage{mathptmx}
% \usepackage[scaled=.90]{helvet}
% \usepackage{courier}
%%%%

% cd /tmp
% wget http://ftp.ktug.or.kr/tex-archive/macros/latex/contrib/misc/cases.sty
% sudo mkdir /usr/share/texmf/tex/latex/cases/
% sudo cp test/cases.sty /usr/share/texmf/tex/latex/cases/
% sudo texhash
%\usepackage{cases}                  % http://ftp.ktug.or.kr/tex-archive/macros/latex/contrib/misc/cases.sty

\usepackage{ifpdf}
\ifpdf
    %%%%%%%%%%%%%%%%%%%%%%%%%%%%%%%%%%%%%%%%%%%%%%%%%%%%%%%
    % PDF compilation with "pdflatex"
    %%%%%%%%%%%%%%%%%%%%%%%%%%%%%%%%%%%%%%%%%%%%%%%%%%%%%%%
    \usepackage[pdftex]{graphicx}
    \def\pdfshellescape{1}
    \pdfcompresslevel=9
    %%%
    % PDF options
    % See http://barrault.free.fr/ressources/rapports/pdflatex/
    \usepackage[pdftex,
        bookmarks = true,
        bookmarksnumbered = true,
        bookmarksopen = true,
        pdfpagemode = UseOutlines,
        pdfstartview = FitH,
        colorlinks,
        citecolor=black,urlcolor=blue,linkcolor=black,
        pdfauthor={Nicolas Bigaouette},
        pdftitle={\mytitle},
        unicode = true,
        plainpages = false,pdfpagelabels
    ]{hyperref}
\else
    %%%%%%%%%%%%%%%%%%%%%%%%%%%%%%%%%%%%%%%%%%%%%%%%%%%%%%%
    % DVI compilation with "latex"
    %%%%%%%%%%%%%%%%%%%%%%%%%%%%%%%%%%%%%%%%%%%%%%%%%%%%%%%
    \usepackage[dvips]{graphicx}
    \newcommand{\url}[1]{{\color{blue}#1}}
\fi

%%%%%%%%%%%%%%%%%%%%%%%%%%%%%%%%%%%%%%%%%%%%%%%%%%%%%%%%%%
%                       Page format
%%%%%%%%%%%%%%%%%%%%%%%%%%%%%%%%%%%%%%%%%%%%%%%%%%%%%%%%%%
\textheight=21cm
\textwidth=17.0cm
\oddsidemargin=0cm
\evensidemargin=0cm
\topmargin=0cm
\headsep=20pt
\topskip=10pt
\large
%%%%%%%%%%%%%%%%%%%%%%%%%%%%%%%%%%%%%%%%%%%%%%%%%%%%%%%%%%

% These adjust how LaTeX puts figures onto pages with text. These values
% reduce the likelihood that a figure will end up by itself on a page.
\renewcommand{\topfraction}{0.85}
\renewcommand{\textfraction}{0.1}
\renewcommand{\floatpagefraction}{0.75}


\renewcommand{\labelenumi}{\alph{enumi}) }

\begin{document}
\maketitle
% \vspace{-50 pts}

\tableofcontents
% \newpage



\section{HS potential}
The HS potential was obtained using the Cowan-code. The resulting points were fitted using these function:
\begin{align}
U\pa{r} & = - \pa{ a   \e{-r b + c} + d   \e{-r e + f} + g   \e{-r h + i} - j } \\
E\pa{r} & = - \pa{ a b \e{-r b + c} + d e \e{-r e + f} + g h \e{-r h + i} }
\end{align}

The parameters are given in tables \ref{table:rmax} and \ref{table:params}.

\begin{table}
\begin{center}
\begin{tabular}{|c|c|} \hline
Charge state    & $r_{max}$ (Bohr)      \\ \hline \hline
Neutral         & 12.0                  \\ \hline
1+              & 3.20671               \\ \hline
2+              & 3.4419260284993465    \\ \hline
3+              & 2.5786383389900931    \\ \hline
4+              & 2.3069998849855851    \\ \hline
5+              & 2.1424931697942897    \\ \hline
6+              & 1.9288708307182563    \\ \hline
\end{tabular}
\end{center}
\caption{\label{table:rmax}Distance where HS potential reaches Coulombic values}
\end{table}


\begin{table}
\hspace{-45pt}
\begin{tabular}{|c|c|c|c|c|c|c|c|} \hline
  & Neutral         & 1+            & 2+            & 3+            & 4+            & 5+            & 6+            \\ \hline \hline
a & 3.14083382      & 6.68943368    & 1.08862692    & 1.26082951    & 8.33659368    & 8.13621709    & 7.52331956    \\ \hline
b & 2.23690529      & 5.46613511    & 1.12845509    & 1.24346292    & 15.53383795   & 15.39455048   & 15.56584267   \\ \hline
c & 2.45999159      & 3.50223818    & 2.40634711    & 2.48202614    & 4.69224278    & 4.6973397     & 4.77821787    \\ \hline
d & 1.04922253      & 9.84854609    & 7.5231977     & 7.60391482    & 5.66740119    & 1.33881001    & 2.17218048    \\ \hline
e & 0.758964055     & 16.33928308   & 11.80857511   & 14.20436211   & 4.93161199    & 1.40783802    & 1.51817071    \\ \hline
f & 0.916259659     & 4.55862051    & 4.40029841    & 4.63976132    & 3.58851214    & 2.72036815    & 2.38100923    \\ \hline
g & 6.43949225      & 2.05998631    & 3.5171341     & 5.75320941    & 1.33122023    & 5.60695758    & 5.09462365    \\ \hline
h & 5.57500308      & 1.79149357    & 4.02105327    & 4.57482337    & 1.36069086    & 4.96351559    & 5.11830058    \\ \hline
i & 3.46700894      & 2.67105113    & 3.70863489    & 3.45935956    & 2.65699251    & 3.59035494    & 3.70739486    \\ \hline
j & -2.22471387e-3  & -0.216514733  & -0.33244088   & -0.55091234   & -0.90941801   & -1.33283627   & -1.84326541   \\ \hline
\end{tabular}
\caption{\label{table:params}HS parameters}
\end{table}

\begin{figure}
\includegraphics[width=0.98\columnwidth]{hs_1}
\caption{\label{fig:hs:1}HS Potential energy and electric field}
\end{figure}

Figure \ref{fig:hs:1} shows the HS potential and field. Notice the discontinuity where it becomes
coulombic. This discontinuity is due to the fact that the even though the potential reaches the
Coulomb values, at this distance $r_{max}$ the derivative, and thus the field, is discontinuous.

\section{Preventing the discontinuity}
Since the important aspect in a Molecular Dynamics code is the force, the field cannot have a
discontinuity without introducing errors. To correct this:
\begin{enumerate}
\item The HS field is multiplied by a constant to shift it upward and match Coulomb at $r_{max}$
\begin{align}
E'\pa{r} & = \frac{\textrm{Coulomb}\pa{r_{max}}}{E\pa{r_{max}}} E\pa{r}
\end{align}
\item This multiplication factor must be applied to the potential too for the field to match the derivative of the potential:
\begin{align}
U'\pa{r} & = \frac{\textrm{Coulomb}\pa{r_{max}}}{E\pa{r_{max}}} U\pa{r}
\end{align}
\item But since the potential energy is now shifted by this multiplication, an additive constant is added so the potential match the Coulomb values at $r_{max}$:
\begin{align}
U''\pa{r} & = U'\pa{r} + \textrm{Coulomb}\pa{r_{max}} - \frac{\textrm{Coulomb}\pa{r_{max}}}{E\pa{r_{max}}} U\pa{r_{max}}
\end{align}
\end{enumerate}

This is performed in function Set\_HermanSkillman\_Lookup\_Tables\_Xe() in HermanSkillman.cpp.




\end{document}


