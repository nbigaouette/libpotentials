%%%%%%%%%%%%%%%%%%%%%%%%%%%%%%%%%%%%%%%%%%%%%%%%%%%%%%%%%%
%%%%%%%%%%%%%%%%%%%%%%%%%%%%%%%%%%%%%%%%%%%%%%%%%%%%%%%%%%
%%          Written by Nicolas Bigaouette               %%
%%                    Winter 2008                       %%
%%              nbigaouette@gmail.com                   %%
%%%%%%%%%%%%%%%%%%%%%%%%%%%%%%%%%%%%%%%%%%%%%%%%%%%%%%%%%%
%%%%%%%%%%%%%%%%%%%%%%%%%%%%%%%%%%%%%%%%%%%%%%%%%%%%%%%%%%

\documentclass[12pt,letterpaper]{article}
% \documentclass[12pt,letterpaper,draft]{article}
\newcommand{\mytitle}{Cubic Spline}
\title{\mytitle}
\date{}
\author{Nicolas Bigaouette}

%%%%%%%%%%%%%%%%%%%%%%%%%%%%%%%%%%%%%%%%%%%%%%%%%%%%%%%
% Useful macros
%%%%%%%%%%%%%%%%%%%%%%%%%%%%%%%%%%%%%%%%%%%%%%%%%%%%%%%
\input{macros.tex}
%%%%%%%%%%%%%%%%%%%%%%%%%%%%%%%%%%%%%%%%%%%%%%%%%%%%%%%


%%%%%%%%%%%%%%%%%%%%%%%%%%%%%%%%%%%%%%%%%%%%%%%%%%%%%%%
% Needed packages
%%%%%%%%%%%%%%%%%%%%%%%%%%%%%%%%%%%%%%%%%%%%%%%%%%%%%%%

% Needed for \begin{subequations}...\end{subequations}
\usepackage{amsmath}
% Needed for bold Greek letters
\usepackage{bm}
% Multiple figures
\usepackage{subfigure}
%%%%%%%%%%%%%%%%%%%%%%%%%%%%%%%%%%%%%%%%%%%%%%%%%%%%%%%
\usepackage{ifthen}
%%%%%%%%%%%%%%%%%%%%%%%%%%%%%%%%%%%%%%%%%%%%%%%%%%%%%%%

%%%%
% Voir http://forums.gentoo.org/viewtopic-t-202973-highlight-fonts+latex.html
% \usepackage{mathptmx}
% \usepackage[scaled=.90]{helvet}
% \usepackage{courier}
%%%%

% cd /tmp
% wget http://ftp.ktug.or.kr/tex-archive/macros/latex/contrib/misc/cases.sty
% sudo mkdir /usr/share/texmf/tex/latex/cases/
% sudo cp test/cases.sty /usr/share/texmf/tex/latex/cases/
% sudo texhash
%\usepackage{cases}                  % http://ftp.ktug.or.kr/tex-archive/macros/latex/contrib/misc/cases.sty

\usepackage{ifpdf}
\ifpdf
    %%%%%%%%%%%%%%%%%%%%%%%%%%%%%%%%%%%%%%%%%%%%%%%%%%%%%%%
    % PDF compilation with "pdflatex"
    %%%%%%%%%%%%%%%%%%%%%%%%%%%%%%%%%%%%%%%%%%%%%%%%%%%%%%%
    \usepackage[pdftex]{graphicx}
    \def\pdfshellescape{1}
    \pdfcompresslevel=9
    %%%
    % PDF options
    % See http://barrault.free.fr/ressources/rapports/pdflatex/
    \usepackage[pdftex,
        bookmarks = true,
        bookmarksnumbered = true,
        bookmarksopen = true,
        pdfpagemode = UseOutlines,
        pdfstartview = FitH,
        colorlinks,
        citecolor=black,urlcolor=blue,linkcolor=black,
        pdfauthor={Nicolas Bigaouette},
        pdftitle={\mytitle},
        unicode = true,
        plainpages = false,pdfpagelabels
    ]{hyperref}
\else
    %%%%%%%%%%%%%%%%%%%%%%%%%%%%%%%%%%%%%%%%%%%%%%%%%%%%%%%
    % DVI compilation with "latex"
    %%%%%%%%%%%%%%%%%%%%%%%%%%%%%%%%%%%%%%%%%%%%%%%%%%%%%%%
    \usepackage[dvips]{graphicx}
    \newcommand{\url}[1]{{\color{blue}#1}}
\fi

%%%%%%%%%%%%%%%%%%%%%%%%%%%%%%%%%%%%%%%%%%%%%%%%%%%%%%%%%%
%                       Page format
%%%%%%%%%%%%%%%%%%%%%%%%%%%%%%%%%%%%%%%%%%%%%%%%%%%%%%%%%%
\textheight=21cm
\textwidth=17.0cm
\oddsidemargin=0cm
\evensidemargin=0cm
\topmargin=0cm
\headsep=20pt
\topskip=10pt
\large
%%%%%%%%%%%%%%%%%%%%%%%%%%%%%%%%%%%%%%%%%%%%%%%%%%%%%%%%%%

% These adjust how LaTeX puts figures onto pages with text. These values
% reduce the likelihood that a figure will end up by itself on a page.
\renewcommand{\topfraction}{0.85}
\renewcommand{\textfraction}{0.1}
\renewcommand{\floatpagefraction}{0.75}


\renewcommand{\labelenumi}{\alph{enumi}) }

\begin{document}
\maketitle
% \vspace{-50 pts}

\tableofcontents
% \newpage

% *******************************************************************
\section{Generic}
For a collection of $n+1$ points, there is $n$ intervals. A cubic spline is a way to calculate
the coefficients of a cubic polynomial on every one of these intervals $i$:
\begin{align}
g\pa{x_i} & = a_{i} + b_{i} \pa{r - r_{i}} + c_{i} \pa{r - r_{i}}^2 + d_{i} \pa{r - r_{i}}^3
\end{align}

For $n+1$ points, we define\footnote{\url{http://people.math.sfu.ca/~stockie/teaching/macm316/notes/splines.pdf}}
the vector $\vh$ of size $n$ as the intervals size:
\begin{align}
h_{i} & = r_{i+1} - r_{i}
\end{align}
A spline will solve:
\begin{align}
\vA \vm & = \vb
\end{align}
A ``natural'' spline will have zero curvature at its boundaries. Such a spline is given by this definition:
\begin{align}
\vA =
\begin{bmatrix}
1       & 0                 &                   &                   &           &                       &                       &   \\
h_{0}   & 2\pa{h_{0}+h_{1}} & h_{1}             & 0                 &           &                       &                       &   \\
0       & h_{1}             & 2\pa{h_{1}+h_{2}} & h_{2}             & 0         &                       & \hdots                &   \\
0       & 0                 & h_{2}             & 2\pa{h_{2}+h_{3}} & h_{3}     & 0                     &                       &   \\
        &                   &                   &                   & \ddots    &                       &                       &   \\
        &                   &                   & 0                 & h_{n-3}   & 2\pa{h_{n-3}+h_{n-2}} & h_{n-2}               & 0 \\
        &                   & \vdots            &                   & 0         & h_{n-2}               & 2\pa{h_{n-2}+h_{n-1}} & h_{n-1} \\
        &                   &                   &                   &           &                       & 0                     & 1
\end{bmatrix}
\end{align}
and:
\begin{align}
\vb = 6
\begin{bmatrix}
0 \\
\frac{y_{2}-y_{1}}{h_{1}} - \frac{y_{1}-y_{0}}{h_{0}}  \\
\frac{y_{3}-y_{2}}{h_{2}} - \frac{y_{2}-y_{1}}{h_{1}}  \\
\vdots \\
\frac{y_{n-1}-y_{n-2}}{h_{n-2}} - \frac{y_{n-2}-y_{n-3}}{h_{n-3}}  \\
\frac{y_{n}  -y_{n-1}}{h_{n-1}} - \frac{y_{n-1}-y_{n-2}}{h_{n-2}}  \\
0
\end{bmatrix}
\end{align}

After solving for $\vm$, the cubic polynomials' coefficients for each sections $i$ is obtained:
\begin{subequations}
\begin{align}
a_i & = y_i \\
b_i & = \frac{y_{i+1}-y_{i}}{h_{i}} - \frac{h_{i} m_{i}}{2} - \frac{h_{i}}{6} \pa{m_{i+1} - m_{i}} \\
c_i & = \frac{m_{i}}{2} \\
d_i & = \frac{m_{i+1} - m_{i}}{6 h_{i}}
\end{align}
\end{subequations}


\section{3 points}
Specializing to $n+1=3$ points (or two intervals $i$), the system to solve:
\begin{align}
\vA \vm & = \vb
\end{align}
becomes
\begin{align}
\vA =
\begin{bmatrix}
1       & 0                 & 0     \\
h_{0}   & 2\pa{h_{0}+h_{1}} & h_{1} \\
0       & 0                 & 1
\end{bmatrix}, \hspace{20pt}
\vb = 6
\begin{bmatrix}
0 \\
\frac{y_{2}-y_{1}}{h_{1}} - \frac{y_{1}-y_{0}}{h_{0}}  \\
0
\end{bmatrix}
\end{align}

We see that $m_{0} = m_{2} = 0$ and that:
\begin{align}
h_{0} m_{0} + 2\pa{h_{0} + h_{1}} m_{1} + h_{1} m_{2} & = 6 \pa{\frac{y_{2}-y_{1}}{h_{1}} - \frac{y_{1}-y_{0}}{h_{0}}} \\
2\pa{h_{0} + h_{1}} m_{1} & = 6 \pa{\frac{y_{2}-y_{1}}{h_{1}} - \frac{y_{1}-y_{0}}{h_{0}}} \\
m_{1} & = \pa{ \frac{3}{h_{0} + h_{1}} } \pa{\frac{y_{2}-y_{1}}{h_{1}} - \frac{y_{1}-y_{0}}{h_{0}}}
\end{align}

Inserting this into the coefficients, we get the coefficients for the first interval:
\begin{subequations}
\begin{align}
a_0 & = y_0 \\
b_0 & = \frac{y_{1}-y_{0}}{h_{0}} - \frac{h_{0} m_{1}}{6} \\
c_0 & = 0 \\
d_0 & = \frac{m_{1}}{6 h_{0}}
\end{align}
\end{subequations}
and for the second interval
\begin{subequations}
\begin{align}
a_1 & = y_1 \\
b_1 & = \frac{y_{2}-y_{1}}{h_{1}} - \frac{h_{1} m_{1}}{2} + \frac{h_{1} m_{1}}{6} \\
c_1 & = \frac{m_{1}}{2} \\
d_1 & = -\frac{m_{1}}{6 h_{1}}
\end{align}
\end{subequations}










\end{document}


